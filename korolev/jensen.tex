\section{Неравенство Йенсена}
\textbf{Theorem 1.1} (Неравенства Йенсена). \emph{Пусть f(x) выпукла вверх на [a, b]. Тогда $\forall{x_{1}}, \ldots, x_{n} \in [a, b]$ и их выпуклой комбинации выполнено неравенство   $\sum^{n}_{k=1} \alpha_{k}f(x_{k}) \leqslant  f(\sum^{n}_{k=1} \alpha_{k} x_{k})$} \\ \\
\emph{Доказательство.}(Докажем по индукции) \\ \\
\textbf{База:}  \emph{n} = 2
\parindent=10mm

Неравентство превращается в определение выпуклой вверх функции, для которой это, очевидно,

выполняется. \\
\textbf{Переход:} Пусть это верно для \emph{n}. Докажем, что это верно для \emph{n} + 1 :
\[ \sum^{n+1}_{k=1} \alpha_{k} = 1, \text{обозначим за} s_{n} = \sum^{n}_{k=1} \alpha_{k} \]

Пусть $\beta_{k} = \dfrac{\alpha_{k}}{s_{n}}.$ Тогда получаем: $\sum^{n}_{k=1} \beta_{k}=1.$

\[\sum^{n+1}_{k=1} \alpha_{k}f(x_{k}) = s_{n} \sum^{n}_{k=1} \beta_{k}f(x_{k}) + \alpha_{n+1}f(x_{n+1}) \leqslant \hspace{65mm} \]
\[\hspace{15mm}\leqslant \text{(по предположению индукции)} s_{n}f\Bigg(\sum^{n}_{k=1} \beta_{k}x_{k}\Bigg) + \alpha_{n+1}f(x_{n+1}) \leqslant\]
\[\hspace{105mm}\leqslant \text{(так как}s_{n} + \alpha_{n+1} = 1) f\Bigg(\sum^{n+1}_{k=1} \alpha_{k}x_{k}\Bigg)\]

Значит, шаг индукции проделан, неравенство доказано для произвольного n.
