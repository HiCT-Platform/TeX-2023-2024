\documentclass{article}
\usepackage{amsmath}
\usepackage{amssymb}
\usepackage{color}
\usepackage[russian]{babel}
\usepackage[utf8]{inputenc}
\usepackage[T2A]{fontenc}
\usepackage[russian]{babel}
\usepackage[utf8]{inputenc}
\usepackage{tikz}
\usepackage{geometry}
\begin{document}
\color{blue}\tableofcontents\color{black}
\newpage
\section{Неравенства Йенсена}
\textbf{Theorem 1.1}(Неравенства Йенсена).\textit{Пусть f(x) выпукла верх на }[a,b]\textit{.Тогда $\forall x_1,....,x_n$ $\in$ }[a,b]\textit{и их выпуклой комбинации выполнено неравнество $\sum_{k -= 1}^n$ $\alpha_k$ f(x$_k$) $\leq$ f($\sum_{k = 1}^n$ $\alpha_k$x$_k$)}\newline\newline
\textit{Доказательство.} (Докажем по индукции)\newline\newline
\textbf{База:} \textit{n} = 2\newline
Неравенство превращается в определение выпуклой вверх функции,для которой это,очевидно,выполняется.\newline
\textbf{Переход:} Пусть это выполняется для \textit{n}.Докажем,что это работает и для \textit{n} + 1\newline
\[\sum_{k = 1}^{n + 1}\alpha_k = 1,\text{обозначим за} ~ s_n = \sum_{k = 1}^{n + 1}\alpha_k\]\newline
Пусть $\beta_k$ = $\frac{\alpha_k}{s_n}$.Тогда получаем:$\sum_{k = 1}^n$$\beta_k$ = 1\newline
\[\mspace{-220mu}\sum_{k = 1}^{n + 1}\alpha_{k}f(x_k) = s_n\sum_{k = 1}^{n}\beta_k f(x_k) + \alpha_{n + 1}f(x_{n + 1}) \leq\]\newline
\[\mspace{110mu}\leq \text{(по предположению индукции)} s_n\left( \sum_{k = 1}^n \beta_k x_k \right) + \alpha_{n + 1} f(x_n + 1) \leq\]\newline
\[\mspace{400mu}\leq (\text{так как} s_n + \alpha_{n + 1} = 1) f\left(\sum_{k = 1}^{n + 1} \alpha_k x_k\right)\]\newline
Значит,наг индукции проделан,неравенство доказано для произвольного n.\newline
\[\mspace{550mu}\square\]
\end{document}