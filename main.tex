\documentclass{article}
\usepackage{imports}
\usepackage[russian]{babel}
\usepackage[14pt]{extsizes}
\usepackage{xcolor}
\usepackage{wasysym}
\usepackage{amsmath}
\usepackage{amssymb}
\title{\bfseries Домашняя работа \textnumero 7 по курсу \TeX'а}
\author{Samsonova Varvara}
\date{\today}
\begin{document}
\maketitle
\tableofcontents 
\newpage
\hline\color{blue}\section{Неравенство Йенсена}

$

\hline\color{black}\textbf{Theorem 1.1}(Неравенство Йенсена). Пусть f(x) выпукла вверх на [a,b]. Тогда \forall x_1,...,x_n \in [a,b] и их выпуклой комбинации выполнено неравенство \sum^n_k=1 \alpha_k f(x_k) \leqslant f(\sum^n_{k=1}\alpha_k x_k)
$

\textit{Доказательство.} (Докажем по индукции)

\textbf{База:}n=2

Неравенство превращается в определение выпуклой вверх функции, для которой это, очевидно выполняется.

\textbf{Переход:} Пусть это верно для n. Докажем, что это верно для n+1:

$
~~~~~~~~~~~~~\displaystyle\sum^{n+1}_{k=1} \alpha_k=1, обозначим зa s_n=\displaystyle\sum^n_{k=1} \alpha_k
пусть \beta_k=\cfrac{\alpha_k}{s_n}. Тогда получаем \sigma^n_{k=1} \beta_k =1

\displaystyle\sum^{n+1}_{k=1} \alpha_k f(x_k)=s_n\displaystyle\sum^n_{k=1} \beta_k f(x_k)+\alpha_n+1 f(x_n+1)  \leqslant (по предположению индукции) s_n f (\displaystyle\sum^n_{k=1} \beta_k x_k)+ \alpha_{n+1} f(x_{n+1}) \leqslant

\leqslant (так как s_n+\alpha_{n+1}=1) f(\displaystyle\sum^{n+1}_{k=0} \alpha_k x_k)
$

Значит, шаг индукции проделан, неравенство доказано для произвольного n.
\newpage
\hline\color{blue}\section{Круги Эйлера}
\hline\color{black}я не поняла как делать круги эйлера, мой мозг больше не способен на умственную работу, прастите
\end{document}
