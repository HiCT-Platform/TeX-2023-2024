\documentclass{article}
\usepackage{amsmath, amssymb}
\usepackage{ucs}
\usepackage[utf8x]{inputenc}
\usepackage[russian]{babel}

\title{Контрольная работа по \LaTeX'у}
\author{Миша Ушаков / Ваня Веряскин / Егор Волков}
\date{\today}

\begin{document}

\maketitle

\tableofcontents

\section{Таблицы}
AAA
\section{задача}
\[\int^1_0 (1-x^2)^n dx\]
\[\begin{array}{c}
     AAA \\
     AAA \\
     
\end{array}
\end{ara}
\end{ara}\]


\section{Определние}
Образом множества S называеться $F(S) = \{b~\in~B~|~\exists~ a~\in~S~b~\in~F(a)~\}$.
Прообразом множества $T$ Называеться $F^{-1}(t)~=~\{a~\in~A~|~F(a)~\bigcap~T~\neq~\varnothing~\}$
\section{Теорема}
\[F(S\bigcap T) = F(S) \bigcap F(T) \Longleftrightarrow F инъективно.\]



\textit{Доказательство:}

Если $F$ не инъективно,то существуют $a_1,a_2$, такие что $F(a_1)~\bigcap~F(a_2)~\neq~\varnothing~$. Положим $S=\{a_1~\}, T=\{a_2~\}$.

Если $F$ инъекция, то $b~\in~F(S~\bigcap~T) \implies \exists a\in S\bigcap T: b \in F(a)$. Но $a \in S \bigcap T$ озночает, что $a \in S, a \in T$, а значит $b \in F(S), b \in F(T) \implies b \in F(S) \bigcap F(T).$

С другой стороны, пусть $b \in F(S) \bigcap F(T).$ Тогда $\exists a_1 \in S: b \in F(a_1), \exists a_2 \in T: b \in F(a_2)$. По определению инъективного соответствия это значит,что $a_1 = a_2 = a \in S \bigcap T \implies b \in F(S \bigcap T)$.
\end{document}
